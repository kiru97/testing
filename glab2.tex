\documentclass{article}
\usepackage{amsmath}
\usepackage{tikz}
\usetikzlibrary{positioning,calc,decorations.pathreplacing,shapes,fit}
\usepackage{colortbl}
\usepackage{booktabs}
\usepackage{url}
\usepackage{pifont}
\newcommand{\cmark}{\ding{51}}%
\newcommand{\xmark}{\ding{55}}%
\usepackage{array, adjustbox,url}
\usepackage{pifont,marvosym} % wasysym
\usepackage{rotating,subfig}
\usepackage{xspace}

\title{GLab 2: Routing}

\author{Christian Grothoff}
\date{KW 49}

\begin{document}
\maketitle

\section{Introduction}

For this lab you will write a set of small programs leading up to the
implementation of an IP router.  While the driver you are given
is written in C, you can use {\em any} language of your choice for the
implementation (as long as you extend the {\tt Makefile} with adequate
build rules).

The basic setup is the same as in Glab 1.

\subsection{ARP}

Implement a program {\tt arp} that:
\begin{enumerate}
\item Watches for ARP queries on the Ethernet link
\item Responds with ARP responses for your own IP address
\item Reads IPv4 addresses from {\tt stdin} (in human-readable format),
  issues ARP queries for those IPv4 addresses and outputs the
  resulting MAC addresses.  The interactive command syntax should be
  ``arp {\em IP-ADDR} {\em IFNAME}'' (i.e. each line is to be prefixed with
  the letters ``arp '', followed by the IPv4 address and the name of
  the network interface).  If the user just enteres ``arp'' without
  an IP address, you should output the ARP table in the format
  ``{\em IP} -$>$ {\em MAC} ({\em IFNAME})'' with one entry per line,
  i.e.
\begin{verbatim}
10.54.25.15 -> 28:c6:3f:1a:0a:bf (eth1)
\end{verbatim}
  (note the leading ``0'' digit in {\tt 0a}).
\item Provides an ARP cache so that it does not have to repeatedly
  make ARP requests to the network for MAC addresses it already knows.
\end{enumerate}
Your programm should be called with the name of the interface, the IP
address\footnote{You may support multiple IPs per network interface,
  using a comma-separated list of IPs and network masks, but this is
  not required.} for that interface and the network mask.  Example:
\begin{verbatim}
$ network-driver eth0 eth1 - \
  arp eth0[IPV4:192.168.0.1/16] eth1[IPV4:10.0.0.3/24]
\end{verbatim}
This means {\tt eth0} is to be bound to 192.168.0.1 (netmask 255.255.0.0)
and {\tt eth1} uses 10.0.0.3 (netmask 255.255.255.0).


\subsection{Routing}

Implement {\tt router} which routes IPv4 packets.
\begin{enumerate}
\item Populate your routing table from the network interface configuration
  given on the command-line using the same syntax as with the {\tt arp}
  program.
\item Use the ARP logic to resolve the target MAC address.  You MUST
  simply drop IP packets for destinations where you do not yet have the
  next hop's MAC address, but you MUST then issue the ARP request to
  obtain the destination's MAC instead (once per dropped IP packet).
\item Make sure to decrement the TTL field and recompute the CRC.
  % add link to logic implementing CRC!
\item Generate ICMP messages for ``no route to host'' and ``TTL exceeded''.
\item Support the syntax {\tt IFC[RO]=MTU} where {\tt MTU} is the
  MTU for IFC.  Example: {\tt eth0=1500}.  Implement and test IPv4 fragmentation
  (including {\em do not fragment}-flag support).
\item Support dynamic updates to the routing table via {\tt stdin}.
  Base your commands on the {\tt ip route} tool.  For example,
  ``route list'' should output the routing table, and
  ``route add 1.2.0.0/16 via 192.168.0.1 dev eth0'' should add
  a route to {\tt 1.2.0.0/16} via the next hop {\tt 192.168.0.1}
  which should be reachable via {\tt eth0}.  Implement at least
  the {\tt route list}, {\tt route add} and {\tt route del} commands.
\end{enumerate}

You do not need to support VLANs, IP multicast or IP broadcast.

\subsection{Suggested testing}

Configure your router with {\tt eth1} using 192.168.0.1/16.  Configure
{\tt eth2} using 10.0.0.1/8 and {\tt eth3} using 172.16.0.1/12.
Connect your notebook as 10.0.0.2 using 10.0.0.1 as the default
route. Set an MTU of 500 on {\tt eth3}.  Set a default route of
192.168.0.2 on the router.

Verify:
\begin{itemize}
\item Correct forwarding? % IP packets flow? Mac updated?
\item Correct address resultion and caching? % ARP cache?
\item Correct IP handling? % IP TTL decremented, ICMP? Checksum?
\item Correct IP fragmentation? % Use eth3 for testing
\end{itemize}

\end{document}
