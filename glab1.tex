\documentclass{article}
\usepackage{amsmath}
\usepackage{tikz}
\usetikzlibrary{positioning,calc,decorations.pathreplacing,shapes,fit}
\usepackage{colortbl}
\usepackage{booktabs}
\usepackage{url}
\usepackage{pifont}
\newcommand{\cmark}{\ding{51}}%
\newcommand{\xmark}{\ding{55}}%
\usepackage{array, adjustbox,url}
\usepackage{pifont,marvosym} % wasysym
\usepackage{rotating,subfig}
\usepackage{xspace}

\title{GLab 1: Switching}

\author{Christian Grothoff}
\date{KW 42}

\begin{document}
\maketitle

\section{Introduction}

For this lab you will write a set of small programs leading up to the
implementation of an Ethernet switch.  While the driver you are given
is written in C, you can use {\em any} language of your choice for the
implementation (as long as you extend the {\tt Makefile} with adequate
build rules).

Your programs should be {\tt executed} by the {\tt network-driver}.
For this, you call
\begin{verbatim}
$ network-driver IFC1 ... IFCn - PROG ARGS
\end{verbatim}
where ``IFC1 ... IFCn'' is the list of interface names that you
want {\tt network-driver} to support (i.e. ``lan0'', ``lan1'')
and ``PROG'' is the name of your binary and ``ARGS'' are the
command-line arguments to ``PROG''.  Note the ``-'' (single minus)
between the last interface name and ``PROG''.  Also, ``PROG'' must
be given with the path or be in the PATH.

Once you start PROG like this, you can read Ethernet frames
and end-user commands from ``stdin'' and write Ethernet
frames (and end-user output) to ``stdout''.

Note that you must follow the {\tt network-driver}'s particular
format for inter-process communication when reading and writing.
You will {\bf not} be communicating directly with the console!

For testing, you will be given access to a USB Ethernet adapter with
four Ethernet interfaces.  You should plug the adapter into the
provided desktop and make your desktop act as a switch or router
between the notebooks of your team.


\section{Grab the instructions}

First, clone the official {\tt skeleton} Git repository for your
course.  Here, you will find the skeleton code for the GLab project.

You can now develop your project with one of three possible setups:
\begin{itemize}
\item Using the 4-port USB adapter. This is the suggested setup, it
  should be basically plug-and-go.
\item Using virtual machines. This allows you to develop your code
  without any special hardware on any GNU/Linux distribution. However,
  it may be more confusing to see what is going on, and you will not
  be able to easily hook up your system to the Netgear for testing
  with VLANs.
\item Using the provided BananaPi R2. The BananaPi R2 has a few
  quirks. The worst one we are aware of right now is that its
  standard Ethernet ports do not properly implement promoscious mode,
  and that it sends out Ethernet frames with the FCS set to zero.
  Both of these issues are problematic for the {\tt switch} and
  {\tt vswitch} projects, but not for the {\tt arp} and {\tt router}
  project.
  
  You can work around these issues by using Ethernet-USB adapters
  instead of the Ethernet ports on the BananaPi R2 (you can and
  should still use one of the Ethernet ports of the BananaPi R2 for
  your management).
\end{itemize}

\subsection{Setup with virtual machines}

Download a basic qemu VM, for example from
\url{https://people.debian.org/~aurel32/qemu/amd64/}.

Then launch your router with four network interfaces and
the respective up scripts and disjoint MACs:
\begin{verbatim}
# qemu-system-x86_64 -hda wheezy.qcov2 \
   -device e1000,netdev=net0,mac=10:10:A0:D0:C0:B1 \
   -netdev socket,id=net0,listen=:10001 \
   -device e1000,netdev=net1,mac=10:10:A0:D0:C0:B2 \
   -netdev socket,id=net0,listen=:10002 \
   -device e1000,netdev=net2,mac=10:10:A0:D0:C0:B3 \
   -netdev socket,id=net0,listen=:10003 \
   -device e1000,netdev=net3,mac=10:10:A0:D0:C0:B4 \
   -netdev socket,id=net0,listen=:10004 
\end{verbatim}
Then, launch additional VMs to connect ``other'' systems
to your router:
\begin{verbatim}
# cp wheezy.qcov2 wheezy0.qcov
# qemu-system-x86_64 -hda wheezy0.qcov2 \
   -device e1000,netdev=net0,mac=10:20:A0:D0:C0:B1 \
   -netdev socket,id=net0,connect:10001
# cp wheezy.qcov2 wheezy1.qcov
# qemu-system-x86_64 -hda wheezy1.qcov2 \
   -device e1000,netdev=net0,mac=10:20:A0:D0:C0:B2 \
   -netdev socket,id=net0,connect:10002
# cp wheezy.qcov2 wheezy2.qcov
# qemu-system-x86_64 -hda wheezy2.qcov2 \
   -device e1000,netdev=net0,mac=10:20:A0:D0:C0:B3 \
   -netdev socket,id=net0,connect:10003
# cp wheezy.qcov2 wheezy3.qcov
# qemu-system-x86_64 -hda wheezy3.qcov2 \
   -device e1000,netdev=net0,mac=10:20:A0:D0:C0:B4 \
   -netdev socket,id=net0,connect:10004
\end{verbatim}
You may want to link up your VMs to your host to copy files:
\begin{verbatim}
# qemu-system-x86_64 ...
  -net nic -net user,hostfwd=tcp::5555-:22
\end{verbatim}
Now your localhost port 5555 is forwarded to port 22 of the
virtual machine.

\subsection{Setup with BananaPi R2}

You must copy the {\tt bpi.img} from the Git repository onto an SD card.
Use
\begin{verbatim}
# dd if=bpi.img of=/dev/sdX
# sync
\end{verbatim}
Replace ``X'' with the letter matching the SD card on your system.

Next, you can put the SD card into the developer board and plug in
the power supply.  You need to hold the ``Power on'' button in the
top left corner of the board for $>$ 10 seconds to turn on the board!

Now you can connect the {\em management} (WAN) Ethernet port to your
notebook or desktop. Your system should use {\tt 192.168.0.1/24}.
You should then be able to log into the development board via
{\tt ssh 192.168.0.2} using the user {\tt root} and the password {\tt bananapi}.

You can either develop the software on the development board itself
(it is powerful enough to run {\tt zile} and {\tt gcc}) or
cross-compile and copy the code over manually.  If you need
additional software on the development board, configure IP forwarding
and masquerading on your desktop system, the provided configuration
already uses {\tt 192.168.0.1} as the default route.

Note that the management (WAN) Ethernet port should only be used
for compiling and configuration, the actual switching should be
done using the four remaining network ports.

\section{Build the driver}

To comple the code, run:
\begin{verbatim}
# This requires gcc
$ make
# Creating network interfaces requires 'root' rights
$ sudo chmod +s network-driver
# Try it out:
$ ./network-driver eth0 - ./parser
\end{verbatim}
Press CTRL-C to stop the {\tt network-driver} and {\tt parser}.



\section{Understanding the driver}

The output of the driver is always in binary and generally in network
byte order.  You can use a tool like {\tt hexer} to make the output
slightly more readable.

The driver will always output a series of messages starting with
a {\tt struct GLAB\_MessageHeader} that includes a type and a size.

When the driver starts, it first writes a control message (of type 0)
with payload that includes 6 bytes for each of the local interface's
MAC addresses to your {\tt stdin}.  Henceforce, messages received
of type 0 will be single lines of command-line input (including the
'\\n'-terminator, but excluding the 0-terminator of C) as typed in
by the user.

Furthermore, the driver will output a {\tt struct GLAB\_MessageHeader}
for each frame received.  The {\tt struct GLAB\_MessageHeader} will be
followed by the actual network frame, starting with the Ethernet frame
excluding preamble, delimiter and FCS.  The {\tt struct
  GLAB\_MessageHeader} includes the total length of the subsequent
frame (encoded in network byte order, the size includes the {\tt
  struct GLAB\_MessageHeader}).  The fixed message type identifies the
number of the network interface, counting from one (also in network
byte order).

In addition to writing received frames to your {\tt stdin}, the driver
also tries to read from your {\tt stdout}.  Applications must send the
same message format to {\tt stdout} that the driver sends them on {\tt
  stdin}.  The driver does {\bf not} check that the source MAC is set
correctly!

To write to the console's {\tt stdout}, use a message type of 0.
You may directly write to {\tt stderr} for error messages.

\subsection{Provided code}

You are given a few C snippets as starting points. However, these
mostly serve to {\em illustrate} how to process the output from the
driver. You are completely free to implement your application in {\em
  any} programming language.  Note that each file includes about 20
LOC of a licensing statement, so the functions provided should not
provide a significant advantage for implementations in C.

\begin{description}
\item[sample-parser.c]{This file includes a simple starting point for
  the parser.  It mostly shows how the frames are received and
  a bit how to use the other C files. (82 LOC)}
\item[glab.h]{A struct defining a MAC Address and a few common C includes. (90 LOC)}
\item[print.c]{This file shows how to wrap messages to print them
  via the driver. (112 LOC)}
\item[loop.c]{This could be the main loop of your application. Includes
  some basic logic to look at each frame, decide whether it is the MACs,
  control or an Ethernet frame and then calls the respective function. (93 LOC)}
\item[crc.c]{An implementation of checksum algorithms. (194 LOC)}
\end{description}

If you are using another programming language, you are free to re-use
an existing CRC implementation in that language.


\subsection{Frame parsing}

Extend the simple frame {\tt parser} to:
\begin{enumerate}
\item Output your system's MAC address(es) in the canonical human-readable format.
\item Output the source MAC, destination MAC, payload type and payload length
  of each frame received. Confirm your results with {\tt wireshark}.
\end{enumerate}

\subsection{Implement a Hub}

Implement {\tt hub} which forwards frames received on any interface to
all other interfaces ({\tt eth0} through {\tt eth3}), without changing
them at all.  The {\tt hub} binary should take the list of interfaces
to use on the command line.

\subsection{Implement a Switch}

Implement {\tt switch} which forwards frames received on any
interface to any other interface, but passively learns MAC addresses
and optimizes subsequent traffic.

\begin{itemize}
\item Change external connections and ensure that switching still works!
\item Create an ``attacker'' process that sends from billions of MAC
  addresses.  Ensure your switch's learning table uses finite memory
  (and ideally constant-time lookups).
\end{itemize}

\subsection{Implement a VSwitch}

Implement {\em vswitch} which forwards frames received on any
interface to any other interface, passively learns MAC addresses,
and respects VLAN tags. As before, the command-line specifies the
list of network interfaces you should switch on, but with
additional options to specify the VLANS.  Example:
\begin{verbatim}
$ network-driver eth0 eth1 eth2 eth3 - \
  vswitch eth0[T:1,2] eth1[U:1] eth2[U:2] eth3[U:2]
\end{verbatim}
This is supposed to run VLANs 1 and 2 tagged on {\tt eth0},
and VLANs 1, 2 or 2 untagged on {\tt eth1}, {\tt eth2},
or {\tt eth3} respectively.  Network interfaces specified
without ``[]'' should operate untagged on VLAN 0.  It is
not allowed to have interfaces accept both tagged and
untagged frames.


\subsection{Suggested testing}

Test your implementation against the Netgear switch of the lab.
Bridge a tagged VLAN ({\tt VID}$=3$) from the Netgear switch ({\tt
  eth1}) with two untagged notebooks ({\tt eth2}, {\tt eth3}) using
the BananaPi.

Evaluation criteria:
\begin{itemize}
\item Correct forwarding? Do frames flow bidirectionally between eth2 and eth3?
\item Correct switching? Is learning correctly implemented?
\item Are VLAN tags added/stripped?
\item Are correct limitations imposed on VLANs?
      Is say $VID=4$ not forwarded to eth2/eth3?
\end{itemize}


\end{document}
